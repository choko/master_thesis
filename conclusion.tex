\chapter*{Conclusion}
\addcontentsline{toc}{chapter}{Conclusion}

Goals of this thesis were to develop a cloud platform for ASR, CloudASR,
  and an annotation interface for annotating speech data.
These goals were successfully accomplished and in several aspects even surpassed
  - in addition to original requirement to create batch recognition mode,
  online speech recognition mode was implemented.
In the following sections all achievements are summarized and
  at the end ideas for future work are proposed.

\section*{Cloud platform for ASR}
The first goal of this thesis was to develop a cloud platform for ASR, CloudASR,
  that would provide API for batch speech recognition mode of the submitted wave files.
This API is similar to Google Speech API,
  which enables users to switch to CloudASR seamlessly.
In addition to that CloudASR provides an API for online speech recognition mode.
The CloudASR comes with a web demo,
  where the users can try out the online speech recognition mode with various languages.
Furthermore, the platform is scalable, customizable and easily deployable.


In terms of scalability,
  the platform is able to run both on single-machine and multi-machine setup
  and it allows to scale number of running workers according to users' needs.
Additionally, the platform can run several API instances and load-balance between them.
The benchmarks show that the platform is able to handle more than 1000 parallel requests
  given enough computational resources.

The platform can handle requests for various languages at the same time.
Moreover, users can create workers for new languages using Pykaldi
  or they can even create workers for an arbitrary ASR systems
  if they provide a Python wrapper for that system.

Finally, CloudASR is easily deployable.
It uses Docker for creating and running application containers.
Therefore, users have to install only Docker for a single machine setup
  and a Mesos cluster for a multi machine setup.
Then it is possible to run the CloudASR platform with just one command.


\section*{Annotation interface}
The second goal of this thesis was to create an annotation interface for annotating submitted recordings.
Its responsibility is to collect and store obtained recordings together with their transcriptions.
Then users can rate transcriptions of the recordings
  or they can add their own transcriptions
  if they think that none is correct.
The annotation interface allows administrators to choose golden transcription from several manual transcriptions
  that were obtained for the recording.
Additionally CloudASR also supports addition of manual transcriptions via external job at CrowdFlower.

\section*{Future work}
\begin{itemize}
  \item
    Since manual transcription of recordings is expensive
      it would be good to make users transcribe only parts of the recordings
      in which ASR system wasn't confident enough \cite{sperber2014fly}.
    This idea could be used for both user transcription and CrowdFlower transcription.

  \item
    With manually transcribed recordings from CloudASR platform
      it is possible to continuously improve accuracy of the underlying ASR system
      by adapting the language model to the type of language that the users of the CloudASR really use.
    Thus CloudASR could provide an option to automatically update language model
      when a certain amount of new transcribed recordings was collected.

  \item
    Because running CloudASR platform is expensive in terms of costs for a server hosting,
      it would be good to optimize usage of individual workers
      so that spare workers are shut down when there is no need for them
      and new workers are started when the traffic arise.
    This can be achieved either by providing feedback control based systems \cite{janert2013feedback}
      or by using machine learning techniques \cite{gong2010press}.

  \item
    As CloudASR platform provides API for speech recognition,
      it could also be used for another speech related tasks like Language Identification, Speaker Identification, Voice Activity Detection, etc.

\end{itemize}


