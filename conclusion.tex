\chapter*{Conclusion}
\addcontentsline{toc}{chapter}{Conclusion}

Goals of this thesis were to develop a cloud platform for ASR, CloudASR,
  and an annotation interface for annotating speech data.
These goals were successfully accomplished and in several aspects even surpassed
  - in addition to original requirement to create batch recognition mode,
  we also implemented online recognition mode.
In the following sections we summarize our achievements in detail and
  at the end we propose ideas for future work.

\section*{Cloud platform for ASR}
The first goal of this thesis was to develop a cloud platform for ASR, CloudASR,
  that would provide batch API for speech recognition of wave files.
The platform uses Master/Worker architecture.
Consequently, it is able to run both on single-machine and multi-machine setup.
The platform allows us to run workers for various language models and to scale workers according to our needs.
To be able to run CloudASR on several machines we chose Mesos/Marathon as an underlying technology
The current implementation of the API supports two modes of speech recognition: batch and online.

Firstly, batch mode allows users to send a file with a recording to the server
  and then it sends transcribed text back as a json.
API of this mode is similiar to Google Speech API
  which allows users to switch from Google Speech API to CloudASR easily.

Secondly, users can transcribe speech recordings in real-time via online mode.
We have also created Python and JavaScript libraries for using our API.


Finally, we wanted CloudASR to be easily deployable.
Because of that, we used Docker for creating and running application containers.
As a result only dependency that users have to install is Docker for single-node setup and Mesos Cluster for multi-node setup.
Moreover, installation scripts for these dependencies are included within the distribution together with deployment scripts,
  that can be used for CloudASR instances management.

\section*{Annotation interface}
The second goal of this thesis was to create an annotation interface for annotating speech data.
First responsibility of the annotation interface is to collect and store obtained recordings.

The second responsibility is to allow users to rate transcriptions of the recordings (Is the transcription correct? yes/no) or to subsequently add their own transcriptions.
The annotation interface implements algorithm to choose golden transcription from several manual transcriptions
  that were obtained for the recording.
Additionally it is also possible to add manual transcriptions via external job at CrowdFlower.

The third responsibility is to provide export of transcribed recordings.
This can be done either by downloading archive from the web or by using Torrent.

\section*{Future work}
\begin{itemize}
  \item
    Since manual transcription of recordings is expensive
      it would be good to make users transcribe only parts of the recordings
      in which ASR system wasn't confident enough \cite{sperber2014fly}.
    This idea could be used for both user transcription and CrowdFlower transcription.

  \item
    With manually transcribed recordings from CloudASR platform
      it is possible to continuously improve accuracy of the underlying ASR system
      by adapting the language model to the type of language that the users of the CloudASR really use.
    Thus CloudASR could provide an option to automatically update language model
      when a certain amount of new transcribed recordings was collected.

  \item
    Because running CloudASR platform is expensive in terms of costs for a server hosting,
      it would be good to optimize usage of individual workers
      so that spare workers are shut down when there is no need for them
      and new workers are started when the traffic arise.
    This can be achieved either by providing feedback control based systems \cite{janert2013feedback}
      or by using machine learning techniques \cite{gong2010press}.

  \item
    As CloudASR platform provides API for speech recognition,
      it could also be used for another speech related tasks like Language Identification, Speaker Identification, Voice Activity Detection, etc.

\end{itemize}


