\chapter*{Introduction}
\addcontentsline{toc}{chapter}{Introduction}

Speech is the most natural form of human communication.
In order to be able to talk with a computer,
  it is crucial to have a good Automatic Speech Recognition (ASR) system.
On one hand, there are several open-source ASR toolkits,
  however deployment of such toolkits requires substantial knowledge,
  which makes them difficult to use for common software developers.
On the other hand, there are a few web services that provide ASR,
  yet these web services do not solve all problems -
  either they are paid, closed-source or they are not customizable.
So \textbf{the first goal of the present thesis is to develop a cloud platform for ASR}
  that is easy to use both from user's and maintainer's point of view.

Although the quality of ASR systems is improving,
  these systems are still far from perfect.
One of the reasons is that the quality of ASR systems depends heavily on the amount of the training data,
  and there is not enough publicly available transcribed speech data for all languages.
By providing free ASR web service it is possible to collect vast amount of recordings
  that can be manually transcribed.
Consequently, \textbf{the second goal of the present thesis is to create an annotation interface}
  so that recordings obtained by CloudASR platform can be annotated and given back to the community.

The following text development and deployment of CloudASR platform and its annotation interface are described.
Chapter~1 introduces Automatic Speech Recognition theory and tools related to CloudASR.
In Chapter~2, tools used for CloudASR development and deployment are presented.
The implementation of CloudASR platform is described in Chapter~3.
Chapter~4 contains results of conducted benchmarks.
Finally, Chapter~5 concludes this thesis.
