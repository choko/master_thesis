\chapter*{Introduction}
\addcontentsline{toc}{chapter}{Introduction}

The most natural form of people's communication is speech.
In order to be able to talk with a computer,
  it is crucial to have a good Automatic Speech Recognition (ASR) system.
On one hand, there are several open-source ASR toolkits,
  however deployment of such toolkits requires substantial knowledge therefore
  for common software developers it's not easy to use them.
On the other hand, there are a few webservices that provide ASR as a service,
  yet these webservices don't solve all problems -
  either they are paid, closed-source or they are not customizable.
So the first goal of the thesis is to develop a cloud platform for ASR
  that is easy to use both from user's and maintainer's point of view.

Although accuracy of ASR systems is improving,
  these systems are still far from perfect.
One of the reasons is that accuracy of ASR systems relies heavily on the amount of the training data
  and there isn't enough publicly available transcribed speech data.
By providing free ASR webservice it is possible to collect vast amount of recordings
  that can be manually transcribed and used later on for further research.
Consequently, the second goal of the thesis is to create an annotation interface
  so that recordings obtained by CloudASR platform can be annotated and given back to the community.

In the following text there will be described development and deployment of CloudASR platform and of its annotation interface.
Chapter 1 introduces ...
In Chapter 2 architecture of CloudASR is described.
Annotation interface and theory related to obtaining of human transcriptions is presented in Chapter 3.
Finally, Chapter 5 concludes this thesis.
User manual and programmer manual can be found in the Attachments.
